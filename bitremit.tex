\begin{refsection}
\chapter{Bitcoin and Remitances: The Potential of `Stupid Phones'}
\label{btc}
In the 2009 whitepaper \parencite{nakamoto2008bitcoin} called \citetitle{nakamoto2008bitcoin} an anonymous individual or collective referred to a Satoshi Nakamoto (the Japanese equivalent of John Smith) describes an online currency called Bitcoin (BTC). 
This currency -the first of a group called cryptocurrencies- has the following economics features:

\begin{itemize}
\item public transactions
\item anonymous addresses
\item fixed amount of units
\item high granularity (1 Bitcoin = 100,000,000 Satoshi)
\item no or low transaction costs
\end{itemize}

An addition to these economic features, the key feature of Bitcoin is decentralisation.
All transactions are kept in a public ledger called the blockchain. An incentive in the form of transaction fees is provided to stimulate the creation of multiple copies of this ledger.
Conflict resolution, in this system, is done through majority consensus of blockchain copy holders.

There are a number of features which make Bitcoin attractive for remittances. Firstly, the low transaction costs are a huge step up from the current facilitators, which charge high percentages. Secondly, government intervention (especially in the recipient country) is prevented by the anonymity. Thirdly, recipients can choose to keep their remittance in a currency which is not controlled by their government. Fourthly, the granularity ensures very small transactions can be conducted.

Bitcoin initially because popular in advanced economies such as those in Europe and North America.
However, since mid 2013 there has been a large shift towards developing economies,
in which comparative advantages of Bitcoin are more pronounced.
For example, since its inception, Mt.Gox (located in Japan) was the largest Bitcoin exchange (marketplace), however in October 2013, Mt.Gox was overtaken by a Chinese exchange.

Bitcoin transactions are conducted by the electronic `signing' of a transaction.
This is a very simple electronic process.
Currently many smartphone apps are available to this purpose.
However, in the targeted developing economies, many phones are not smartphones, the vast majority using Nokia Operating System (Nokia OS).
 These phone can also run applications, using the Java Mobile App platform. Currently there are no Bitcoin apps available for this platform.

I propose to participate in the development of open-source applications which would enable the users of stupid phones.
We can then exploit the temporal difference when releasing the applications in different languages at different times,
as well the downloads from traceable IP addresses.

\section{Data}
\label{btc:data}
Funds held at Bitcoin exchanges are not held in the owners bitcoin wallet.
Rather, the exchange holds them in their wallet, and the clients have a claim on this wallet.
Much like the way bank clients have a claim to their money, which is in the bank's vault.

Therefore Bitcoin purchases at exchanges do not lead to transaction being recorded in the blockchain.
Since the Bitcoins remain in the Exchange's wallet, the only thing which changes is the claims of the clients.

\begin{itemize}
 \item The blockchain (public ledger)
 \begin{itemize}
 	\item IP address
 	\item Language of transaction notice
 	\item Transaction times
 \end{itemize}
 \item The exchanges
 \begin{itemize}
 	\item Location of exchanges
 	\item Transaction costs for each currency
	\item The order books
	\item Transaction times
 	\item The traded currency
 \end{itemize}
 \item Software
 \begin{itemize}
 	\item Downloads
 	\item Translation contributions (?)
 	\item Connections to server
 \end{itemize}
\end{itemize}

\printbibliography
\end{refsection}