\begin{refsection}
\chapter{End notes}
\label{end-notes}
In the previous chapters I have presented my current research projects.
These projects are in various stages of completions.


The South Africa paper provides an interesting opportunity to exploit the data from the household survey,
done before and after a highly relevant policy change.
Especially since the methodology and location are the same as in \textcite{duflo2000child, duflo2003grandmothers}.
We find results that are highly relevant for the optimal design of cash transfer schemes.
At this point, the analysis is troubled by the non establishment of the common trend, prior to the treatment.
Using additional data or the control group this can be remedied, which will make the results more relevant still.

\textcite{morris1998unique} is analysed in detail.
The main issue is the conflation of risk (quantifiable) and Knightian uncertainty (unquantifiable).
However, there are two more issues in this paper which I also discuss, each of these invalidate the results.
However, I would like to present a worked out example which demonstrates how this convolution leads to the main result of the paper.
In addition to this, this example should demonstrate how only a part of the `ripe for attack' region is solved for the unique equilibrium.

Relating to Bitcoin \parencite{nakamoto2008bitcoin}, I present two ideas.
I think that there is a enormous potential for uncontrolled low-cost remittances using Bitcoin.
I would like to analyse the spread of this.
I propose to follow the spread of applications for platforms which are popular in developing economies.

Lastly, I present a proposal for a theoretical model on how cryptocurrencies gain momentum, this model can be extended to address two issues relating to Bitcoin, namely the supposed non-inflation, as well as the unlimited socially wasteful coin mining.

Here I present two more ideas which are too preliminary to include in the main body of the text.

Firstly, I was in charge of setting up a baseline diagnostic into security perceptions in Conakry, Guinea. This diagnostic will be used to analyse the impact of a police reform conducted here. The reform constitutes of the implementation of a model of `police de proximitee'. The baseline consist of 4500 interviews using smartphones in the commune of Conakry as well a large number of interviews in N'Zerekore (east Guinea). Some time after the implementation of the police reform as follow up study will be conducted. I propose to conduct an impact evaluation using the results from the studies in which I have taken and will take part.

Secondly, I have previously studied philosophy of science in my bachelors in Theoretical Philosophy. I would like to write one paper about an issue relating to this. In particular I would like to at the effect of outliers. The effect how these outliers has extensively been discussed in contexts such as finance and international economics \parencite{taleb2010black,   sornette2009dragon},  as well as individual behaviour \parencite{kahneman2011thinking}. I would like to see how these principles affect research in development microeconomics.

In this document I have tried to give a description of my research projects. None of these projects are near completion and many are in fact in a very early stage. The purpose of this has been to describe the direction I am taking with my thesis, and to receive feedback on this. As a result of this being work in progress, the articles are still incomplete, this includes lacking the proper attribution of ideas using citations.

\printbibliography
\end{refsection}