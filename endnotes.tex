\begin{refsection}
\chapter{End notes}
\label{end-notes}
In the previous chapters I have presented my current research projects.
These projects are in various stages of completions.

I believe that my South Africa paper provides an interesting opportunity to exploit the data from the household survey,
done before and after a highly relevant policy change.
Especially since the methodology and location are the same as in \textcite{duflo2000child, duflo2003grandmothers}.
In addition to this there is now a third data wave which has become available.
This provides us with more data, and mostly the opportunity to use more interesting methodology.
The text of the paper still has to be written. 

% The replication of Surviving Andersonville is already completed in term of the replication in Stata.
% We would like to redo the analysis in R \parencite{R}.
% The relevance of this for replications can be seen from the recent replication \textcite{bell2013questioning} of the original \textcite{rauchhaus2009evaluating} paper.
% The findings of this replication can summarised as:

%\begin{quote}
%``We were easily able to replicate Rauchhaus’ key findings in Stata, but couldn't get it to work in R. It took us a long while to work out why, but the reason turned out to be an error in Stata: Stata was finding a solution when it shouldn’t'have (because of separation in the data). This solution, as we show in the paper, was wrong – and led Rauchhaus’ paper to overestimate the effect of nuclear weapons on conflict by a factor of several million.''~\parencite{bell2013questioning}
%\end{quote}

% We do not expect to find anything similar to Bell and Miller,
% but think that it can nevertheless be interesting to replicate findings using a different software package.
% In R, we can replicate this using the package called `survival', which supports the Kaplan Meier and NPMLE methods.
% In addition to this, the paper needs to be rewritten to emphasise the key finding of our replication.

The discussion of \textcite{morris1998unique} is in a very early stage.
I have looked at the paper in detail, I have also done a brief analysis of my main critique,
namely the conflation of risk (quantifiable) and Knightian uncertainy (unquantifiable).
However, I would like to presenta worked out example which demonstrates how this convolution leads to the main result of the paper.
In addition to this, this example should demonstrate how only a part of the `ripe for attack' region is solved for the unique equilibrium.

Finally, relating to Bitcoin \parencite{nakamoto2008bitcoin}, I think that there is a enormous potention for uncontrolled low-cost remintances using Bitcoin. I would like to analyse the spread of this. I propose to follow the spread of applications for platforms which are popular in developing economies.

At this point, three out of four of the previously discussed research projects are some form of a replication of prior work. As replications they are mostly critiques of others' work, rather than constructive ideas. This is mostly a result of the fact that these projects progress faster. For this reason, these research projects have progressed more than the projects which are primarily based on my own ideas. However, it is my attention to shift the focus of my thesis to include more original work. Here I present two more ideas which are too preliminary to include in the main body of the text, but are based on my own work.

Firstly, I was in charge of setting up a baseline diagnostic into security perceptions in Conakry, Guinea. This diagnostic will be used to analyse the impact of a police reform conducted here. The reform constitutes of the implementation of a model of `police de proximitee'. The baseline consist of 4500 interviews using smartphones in the comune of Conkary as well a large number of interviews in N'Zerekore (east Guinea). Some time after the implementation of the police reform as follow up study will be conducted. I propose to conduct an impact evaluation using the results from the studies in which I have taken and will take part.

Secondly, I have previously studied philosophy of science in my bachelors in Theoretical Philosophy. I would like to write one paper about an issue relating to this. In particular I would like to at the effect of outliers. The effect how these outliers has extensively been discussed in contexts such as finance and international economics \parencite{taleb2010black,   sornette2009dragon},  as well as indiviual behaviour \parencite{kahneman2011thinking}. I would like to see how these principles affect research in development microeconomics.

In this document I have tried to give a description of my research projects. None of these projects are near completion and many are in fact in a very early stage. The purpose of this has been to describe the direction I am taking with my thesis, and to receive feedback on this. As a result of this being work in progress, the articles are still incomplete, this includes lacking the proper attribution of ideas using citations.

\printbibliography
\end{refsection}