\begin{refsection}
\chapter{Introduction}
\label{intro}
In this Preliminary Thesis Defence I present the research projects that I intend to use for my PhD dissertation. These projects are titled:

\begin{itemize}
  \item Additional South African Evidence on Pensions and Child Growth (\autoref{sa})
  \item Risk and Uncertainty in Unique Equilibria in Currency Attacks (\autoref{unc})
  \item Bitcoin and Remittances: The Potential of `Stupid Phones' (\autoref{bitremit})
  \item Bitcoin Attacks: Or How Cryptocurrencies Gain Momentum (\autoref{bitatt})
\end{itemize}

These project are in various stages of completion.
Below I will briefly describe the research idea, the current state, and the potential for each project.

Firstly, the paper Additional South African Evidence of Pensions and Child Growth builds upon \textcite{duflo2000child, duflo2003grandmothers}.
These papers look at the effect of the gender of grandparents on the growth metrics of their grandchildren.
A problem in this comparison lies in the fact that pension eligibility of men was at 65 years, whereas it was 60 for women.
This paper makes use of a Household Survey from 2008, 2012 and 2013. The 
survey is especially of interest, because the pension eligibility age for men was lowered from 65 to 60 in 2009.
I have analysed this data using a Difference-in-Differences approach.
I will discuss this in \autoref{sa}.

\textcite{morris1998unique} claims that introducing a measure of risk in a model of currency attacks, leads to a unique equilibrium.
It thus solves the previous multiple-equilibrium zone called `ripe for attack' for a unique equilibrium.
I believe that this is a result of the fact that the risk factor introduces replaces uncertainty, not certainty.
We thus gain information.
I discuss this paper in light of the distinction between Knightian uncertainty and quantified risk,
and find there to be three major issues in the paper, each of which leads to the results not being replicable.
This is done in \autoref{unc}.

Bitcoin is the first of a new breed of currencies, commonly referred to as a cryptocurrencies. 
Bitcoin transactions are conducted using an electronic process called signing.
I would like to study the role of Bitcoin in remittances.
For this I would like to participate in building Bitcoin applications for the type of simple phones common in developing economies (sometimes referred to a stupid phones).
We could study the spread of Bitcoin by looking at the downloads of the apps from regions,
as well looking at how releasing the app in different languages evolves.
This is discussed in \autoref{bitremit}

Lastly, I present a number of idea for modelling Bitcoin and alternative cryptocurrencies.
These idea include, a model on how a cryptocurrency gains momentum,
a model on how multiple models can gain momentum,
and finally a model which describes a multi cryptocurrency economy.
This last model will also enable us to address to issues about cryptocurrencies.
Firstly, about how there supposedly there is no possibility for inflation,
and secondly, how eventually an enormous amount of resources will be directed at the socially wasteful cryptocurrency mining.
I discuss this idea in \autoref{bitatt}.

After this the papers follow, in the order listed above.
A separate citations section is included for each chapter.
For ease of reference, the titles of the citations in the bibliography link to the online articles using DOIs and URLs.
In the end notes (\autoref{end-notes}) I present a brief discussion of my dissertation, as it stands now, a whole.
I also mention some very preliminary ideas.

\printbibliography
\end{refsection}