\begin{refsection}
\chapter{Introduction}
\label{intro}
In this Preliminary Thesis Defence I present the research projects that I intend to use for my PhD dissertation. These projects are titled:

\begin{itemize}
\item Additional South African Evidence on Pensions and Child Growth (\autoref{sa})
% \item Replication: Surviving Andersonville (\ref{andersonville})
\item Risk and Uncertainty in Unique Equilibria in Currency Attacks (\autoref{unc})
\item Bitcoin and Remittances: The Potential of `Stupid Phones' (\autoref{bitremit})
\end{itemize}

These project are in various stages of completion.
Below I will briefly describe the research idea, the current state, and the potential for each project.

Firstly, the paper Additional South African Evidence of Pensions and Child Growth builds upon \textcite{duflo2000child, duflo2003grandmothers}.
These papers look at the effect of the gender of grandparents on the growth metrics of their grandchildren.
A problem in this comparison lies in the fact that pension eligibility of men was at 65 years, whereas it was 60 for women.
This paper makes use of a Household Survey from 2008, 2012 and 2013. The 
urvey is especially of interest, because the pension eligibility age for men was lowered from 65 to 60 in 2009.
As it stands now I have analysed the data from 2008 and 2012.
The 2013 data was released several weeks ago, I will incorporate these data in future versions of my analysis.
I will discuss this in \autoref{sa}.

% Furthermore, in \autoref{andersonville},
% together with several other PhD students we have replicated the paper ``\citetitle{costa2007surviving}'' by \textcite{costa2007surviving}.
% We were able to replicate all results from the paper, save for a few decimal numbers,
% which we have taken to be the result of rounding errors.

% However, replicating these numbers required us to take a number of unorthodox steps in our analysis.
% When in stead, we take more orthodox steps, we find that we cannot quantitatively and qualitatively replicate the authors' results.
% As an example, we find that the dataset distinguishes between two different POW camps both located in the town of Andersonville.
% The authors however, aggregate these two.
% This leads to overstated magnitudes of the social ties of POWs.
% This is problematic because the main finding of the paper is that the size of these social networks is relevant to an individual's odds of survival.
% At this point our replication is quite complete.
% Before rewriting this paper, I will redo the analysis in R \parencite{R}, currently the original and our replication are in Stata.

\textcite{morris1998unique} claims that introducing a measure of risk in a model of currency attacks, leads to a unique equilibrium.
It thus solves the previous multiple-equilibrium zone called `ripe for attack' for a unique equilibrium.
I believe that this is a result of the fact that the risk factor introduces replaces uncertainty, not certainty.
We thus gain information.
I want to discuss MS1998 in light of the distinction between Knightian uncertainty and quantified risk.
This is done in \autoref{unc}.

Lastly, Bitcoin is the first of a new breed of currencies, commonly referred to as a cryptocurrencies. 
Bitcoin transactions are conducted using an electronic process called signing.
I would like to study the role of Bitcoin in remittances.
For this I would like to participate in building Bitcoin applications for the type of simple phones common in developing economies (sometimes referred to a stupid phones).
We could study the spread of bitcoin by looking at the downloads of the apps from regions,
as well looking at how releasing the app in different languages evolves.

After this the papers follow, in the order listed above.
A separate citations section is included for each chapter.
For ease of reference, the titles of the citations in the bibliography link to the online articles using DOIs and URLs.
In the end notes (\autoref{end-notes}) I present a brief discussion of my dissertation, as it stands now, a whole.
I also mention some very preliminary ideas.

Recently there has been a shift towards more replication and further analysis of previously published results. E.g. \textcite{ioannidis2005most}.

\printbibliography
\end{refsection}