\documentclass[a4paper]{report}
\usepackage{color}
\usepackage{mathtools}
\usepackage{rotating}
\usepackage{accents}
\usepackage{graphicx}
\usepackage[authordate,noibid,backend=biber]{biblatex-chicago}
\usepackage[colorlinks=true,linkcolor=black,citecolor=black,urlcolor=black]{hyperref}

\title{Preliminary Thesis Defence\\~\\
\begin{tabular}{rl}
Supervisor:&Jean-Louis Arcand\footnote{Professor of Economics, The Graduate Institute, Geneva; jean-louis.arcand@graduateinstitute.ch}\\
Second Reader:&Ugo Panizza\footnote{Professor of Economics, The Graduate Institute, Geneva; ugo.panizza@graduateinstitute.ch}
\end{tabular}
}

\author{Bastiaan Quast\thanks{PhD Student, The Graduate Institute, Geneva; bastiaan.quast@graduateinstitute.ch / bquast@gmail.com}\\~\\
The Graduate Institute, Geneva}

\let\oldmarginpar\marginpar
\renewcommand\marginpar[1]{\-\oldmarginpar[\raggedleft\footnotesize #1]
{\raggedright\footnotesize #1}}

\addglobalbib{bibliography.bib}

\newbibmacro{string+doiurlisbn}[1]{%
  \iffieldundef{doi}{%
    \iffieldundef{url}{%
      \iffieldundef{isbn}{%
        \iffieldundef{issn}{%
          #1%
        }{%
          \href{http://books.google.com/books?vid=ISSN\thefield{issn}}{#1}%
        }%
      }{%
        \href{http://books.google.com/books?vid=ISBN\thefield{isbn}}{#1}%
      }%
    }{%
      \href{\thefield{url}}{#1}%
    }%
  }{%
    \href{http://dx.doi.org/\thefield{doi}}{#1}%
  }%
}

\DeclareFieldFormat{title}{\usebibmacro{string+doiurlisbn}{\mkbibemph{#1}}}
\DeclareFieldFormat[article,incollection]{title}%
    {\usebibmacro{string+doiurlisbn}{\mkbibquote{#1}}}
\IfFileExists{upquote.sty}{\usepackage{upquote}}{} 

\begin{document}

\maketitle
\tableofcontents
% \listoftables
% \listoffigures

\begin{refsection}
\chapter{Introduction}
\label{intro}
In this Preliminary Thesis Defence I present the research projects that I intend to use for my PhD dissertation. These projects are titled:

\begin{itemize}
\item Additional South African Evidence on Pensions and Child Growth (\autoref{sa})
% \item Replication: Surviving Andersonville (\ref{andersonville})
\item Risk and Uncertainty in Unique Equilibria in Currency Attacks (\autoref{unc})
\item Bitcoin and Remittances: The Potential of `Stupid Phones' (\autoref{btc})
\end{itemize}

These project are in various stages of completion.
Below I will briefly describe the research idea, the current state, and the potential for each project.

Firstly, the paper Additional South African Evidence of Pensions and Child Growth builds upon \textcite{duflo2000child, duflo2003grandmothers}.
These papers look at the effect of the gender of grandparents on the growth metrics of their grandchildren.
A problem in this comparison lies in the fact that pension eligibility of men was at 65 years, whereas it was 60 for women.
This paper makes use of a Household Survey from 2008, 2012 and 2013. The 
urvey is especially of interest, because the pension eligibility age for men was lowered from 65 to 60 in 2009.
As it stands now I have analysed the data from 2008 and 2012.
The 2013 data was released several weeks ago, I will incorporate these data in future versions of my analysis.
I will discuss this in \autoref{sa}.

% Furthermore, in \autoref{andersonville},
% together with several other PhD students we have replicated the paper ``\citetitle{costa2007surviving}'' by \textcite{costa2007surviving}.
% We were able to replicate all results from the paper, save for a few decimal numbers,
% which we have taken to be the result of rounding errors.

% However, replicating these numbers required us to take a number of unorthodox steps in our analysis.
% When in stead, we take more orthodox steps, we find that we cannot quantitatively and qualitatively replicate the authors' results.
% As an example, we find that the dataset distinguishes between two different POW camps both located in the town of Andersonville.
% The authors however, aggregate these two.
% This leads to overstated magnitudes of the social ties of POWs.
% This is problematic because the main finding of the paper is that the size of these social networks is relevant to an individual's odds of survival.
% At this point our replication is quite complete.
% Before rewriting this paper, I will redo the analysis in R \parencite{R}, currently the original and our replication are in Stata.

\textcite{morris1998unique} claims that introducing a measure of risk in a model of currency attacks, leads to a unique equilibrium.
It thus solves the previous multiple-equilibrium zone called `ripe for attack' for a unique equilibrium.
I believe that this is a result of the fact that the risk factor introduces replaces uncertainty, not certainty.
We thus gain information.
I want to discuss MS1998 in light of the distinction between Knightian uncertainty and quantified risk.
This is done in \autoref{unc}.

Lastly, Bitcoin is the first of a new breed of currencies, commonly referred to as a cryptocurrencies. 
Bitcoin transactions are conducted using an electronic process called signing.
I would like to study the role of Bitcoin in remittances.
For this I would like to participate in building Bitcoin applications for the type of simple phones common in developing economies (sometimes referred to a stupid phones).
We could study the spread of bitcoin by looking at the downloads of the apps from regions,
as well looking at how releasing the app in different languages evolves.

After this the papers follow, in the order listed above.
A separate citations section is included for each chapter.
For ease of reference, the titles of the citations in the bibliography link to the online articles using DOIs and URLs.
In the end notes (\autoref{end-notes}) I present a brief discussion of my dissertation, as it stands now, a whole.
I also mention some very preliminary ideas.

\printbibliography
\end{refsection}

\begin{refsection}
\chapter{Pensions and Child Growth in South Africa}
\label{sa}
\section*{Abstract}
In this paper I look at differences in child malnutrition levels depending on the gender of the recipient of income.

I do this by comparing z-scores of anthropometrics of South African children living in the same household as state pension recipients.
This paper exploits the fact that the data set consists of two surveys,
which were done before and after the lowering of the pension eligibility age for men to the same age as women.

The main preliminary finding is that the household effect was only significant in 2012.

\section{Introduction}
\label{sa:intro}
This paper looks at the effect of the gender of pension recipients on the growth of children in their household in South Africa.
The approach is very similar to \textcite{duflo2000child,duflo2003grandmothers}.
The difference from international standards \parencite[WHO Child Growth Standards]{who2006child} for weight-for-height and length-for-age are computed as z-scores.
These are then compared for different pension recipient status.

The anthropometrics are useful for computing z-scores. These z-scores are considered a good representation of short-term or long-term malnutrition respectively, especially for children between 6 and 60 months old.

The South African pension is a useful variable to measure income because of its criteria.
Besides a maximum level of income, the only criterium is the age of a person.
Because of this the status as a recipient is quite exogenous and there are few selection bias problems.
The problematic difference between the eligibility age of men and women was eliminated between the two surveys which creates an interesting natural experiment.

This study deviates from the Duflo study in several ways.
First of all, the data from the \textcite{saldru2008nids,saldru2012nids} surveys, contains actual information on income,
including pension recipient status.
Whereas Duflo uses age as a proxy for recipient status.
Secondly, between mid 2008 and the end of 2009 the pension age for men was gradualy lowered to sixty, which is at par with women.
Another deviation is the usage of \textcite[WHO Child Growth Standards]{who2006child} in stead of \textcite[CDC Growth Charts: United States]{nchs2000cdc}.
Since these have superseded the CDC charts, however this should not be of any consequence.

The main preliminary result is that the household effect (i.e. having one or more recipients in the household),
is very insignificant in the 2008 estimate.
But in the 2012 estimate it is very significant.

\section{Data}
\label{sa:data}
In this paper I use data from two sources.
The first is the South African National Income Dynamics Survey \parencite[NIDS]{saldru2008nids, saldru2012nids, saldru2013nids} and the second is the World Health Organization's Child Growth Standards \parencite[WHO]{who2006child}.

The main source of data is the NIDS.
This survey collects data on a representative set of appproximately 10,000 South-African households.
The primary information types I use are, the child anthropometrics, the age and gender of household members, and the status as state pension recipients.

For adults several variables measure the different amounts and sources of income.
Among those, a variable if the adult receives a state pension, and if so, how much.
This is a numeric variable, the values of which lie very close together.
For simplicity I have temporarily used this variable as a dummy.

Children's anthropometrics are taken, these are length/height, weight, and waist.
Using these anthropometics and WHO growth standards, z-scores have been calculated.
Unfortunately wave 2 (2012) accidentally omitted the z-scores, so that these cannot be evaluated until an updated version is published.
However, I have computed the length-for-age z-scores manually.

In 2006 the WHO published its standards for child growth \parencite{who2006child}.
These standards are based on the scores of children from different ethnic populations in households which observed a healthy lifestyle.
The standards provide the means and standards deviations used. 
These are on a monthly basis for height-for-age, and on a semi-centimeter level for weight-for-height scores.

\section{Analysis}
\label{sa:analysis}
I perform three types of analysis.
The first analysis uses the Duflo method, whereby I compute the z-scores (using WHO in stead of CDC, this is of no consequence)
and then compare based on living in the household with an of-pension-eligible adult.
Secondly, I use my computed z-scores and the data set provided variable which describes the actual pension received (or lack thereof).
Lastly, I use this state pension variable to evaluate the z-scores provided in the dataset.

The first type of analysis is almost identical to Duflo.
Using an internation set of standard anthropometrics for healthy children between 6 and 60 months old, I standarize observed anthropometrics.
I then construct a number of dummies of different ages and gender of the pension recipients.

In the second analysis, I use the same z-scores I calculated for the Duflo method.
However, here I use the state pension recipient variable provided by the NIDS data set.
For simplicity I change the numerical variable to a dummy variable (as the benefits are virtually identical this is without much loss of information).

This method uses the pre calculated z-scores from the data set and the state pension dummy.
Unfortunately the 2012 z-scores are missing except for height-for-age.
I therefore use only the height-for-age z-scores for the 2012 estimates.

In all of these methods the z-scores can be compared on a number of properties.
Firstly, the status of living in the same household as a state pension recipient or not is the first property to compare.
Secondly, when living with a recipient,the effects of male and female recipients can be compared.
Thirdly, the effects based on the gender of the child can be compared.

\printbibliography
\end{refsection}

%%%%%%%%%%%%%%%%%%%%%%%%%%%%%%%%%%%%%%%%%%%%%%%%%%%%%%%
% UNCOMMENT TO INCLUDE THE ANDERSONVILLE CHAPTER!!!!!
%%%%%%%%%%%%%%%%%%%%%%%%%%%%%%%%%%%%%%%%%%%%%%%%%%%%%%%
% \input{andersonville.tex}

\begin{refsection}
\chapter{Risk and Uncertainty in Unique Equilibria in Currency Attacks}
\label{unc}

\section*{Abstract}
\textcite{morris1998unique} finds that the introduction of a risk factor leads to a unique equilibrium in their model of self-fulfilling currency attacks. The currency attack model thus has the surprising feature, with the introduction of perception distortion, the multiple equilibria region seems to solve to a single equilibrium. It is our contention that this is not a surpising result, it a consequence of the second-order effect, where the risk factor replaces a previous existent, Knightian uncertainty. Previously, speculators had no expectations on other speculators expectations. Here, the risk factor replaces uncertainty, this is the main driver of the surprising result of the paper. Also, the new quantified-risk region only solves a certain part of the multiple equilibria region, not the whole of it. Lastly, if we replace the somewhat strange assumption of a bounded uniform distribution of risk perceptions, with the more orthodox, normal distribution, we cannot derive the same result.

\section{The Paper}
\label{unc:idea}
\textcite{morris1998unique} addresses the issue of multiple equilibria in models of currency attacks.
In situations of a currency peg, there is the possibility of a currency attack.
Speculators are able to short the pegged currency, hoping the government will release the peg.
If a sufficiently large proportion of the market participates in this,
the cost of maintaining the peg for the government becomes too high,
which will lead to the government releasing the peg.
Currency attacks thus have a self-fulling nature,
which leads to a situation of multiple equilibria,
as described in \textcite{obstfeld1986rational,obstfeld1995logic,obstfeld1996models}.

In \textcite{morris1998unique} the authors describe such a model.
This model is characterised by a situation where there is a stable, unstable,
and a `ripe for attack' region (based on the underlying economic fundamentals).
The authors expand on the standard model by introducting second-order expectations.
Hereby speculators do not only look at the economics fundamentals of a currency,
but also at other speculators' perceptions of these fundamentals.

The authors proceed by introducing a measure of risk around perceptions of the fundamentals by investors.
Investors do not know the actual state of the fundamentals, but rather a distored form of it.
As a result, the investors' perception of other investors' perceptions are even more distorted.
Paradoxically, the leads to model being solvable to a unique equilibrium.

\section{The Model}
\label{unc:model}
I will give a brief description of the model as it is defined, and some it features.
There is a state of economic fundamentals $\theta$ which is distributed as $\theta \sim U[0,1]$.
The exchange rate is the hypothetical situation of no government intervention is a function only of these fundamental ($f(\theta)$),
it is assumed that $\frac{\partial f}{\partial \theta} > 0$.
The currency is pegged at a level larger than that derived from the fundamentals ($e* \geq f(\theta)$).
Speculators can short the currency at a cost $t$, their payoff is described by:
\begin{equation}\label{unc:specPO}
e^* - f(\theta) - t
\end{equation}
The government derives a positive value $\nu > 0$ from defending the peg.
The cost of the peg is determined by the economic fundamentals and the proportion of the speculators ($\alpha$) that attack the currency.
The payoff the government is thus:
\begin{equation}\label{unc:govPO}
\nu - c(\alpha, \theta)
\end{equation}
Lastly, the following, assumptions are imposed: $c(0,0) > \nu$ (with worst fundamtentals the peg always has to be released),
$c(1,1) > \nu$ (with all speculators attacking, the peg always has to be released),
$e^* - f(1) < t$ (with best fundamentals, it is inopportune for speculators to attack).

\subsection{The Possible Outcomes}
When we set \autoref{unc:specPO} and \autoref{unc:govPO} equal to zero,
we derive the two turning points in the model.
We define $\underaccent{\bar}{\theta}$ which solves $c(0, \theta) = \nu$.
Also, $\bar{\theta}$ solves $f(\theta) = e^* - t$.

Using the turning points described above, we can define three possible outcome intervals:
\begin{enumerate}
	\item $[0, \underaccent{\bar}{\theta}]$, the cost is always too high, this is the unstable region.
	\item $[\underaccent{\bar}{\theta}, \bar{\theta}]$, if enough speculators attack, the cost of defending the peg becomes too high,
	this is the `ripe for attack' region.
	\item $[\bar{\theta}, 1]$, the cost of shorting the currency will always outway the possible gains,
	this is the stable region.
\end{enumerate}
The two corner intervals have unique equilibria, however, the middle interval, does not. This is thus the multiple equilibria region.


\section{The Introduction of Risk}
The authors expand the model by introducing a measure of risk for fundamentals:
\begin{equation}
x \sim U[\theta - \epsilon, \theta + \epsilon]
\end{equation}
Where is the $x$ is the state of the fundamentas \textit{observed} by the speculators.
A speculators observation can thus \textit{at most} deviate from the true value of the fundamentals by $\epsilon$.
The speculators are also aware of the nature of the distortion.
This means that they are also aware of the fact that other speculators' perceived value of the fundatmentals can at most,
deviate from their own with a magnitude of $2\epsilon$.

\subsection{The Outcomes with Risk}
\label{unc:results}
Having established the new model, the authors state:
\begin{quotation}
	The unique optimal strategy for the government is then to abandon the exchange rate only if the observed fraction of deviators, $\alpha$, is greater than or equal to the critical mass $a(\theta)$ in the prevailing state $\theta$.\hspace{\stretch{1}}\parencite[p.~591]{morris1998unique}
\end{quotation}
By solving out this government `optimal' strategy,
the speculators can ensure themselves of a certain abandonment of the peg,
for a sufficiently low observed $x$.
The authors proceed to derivde from this the main result of the paper, which is:
\begin{quotation}
	THEOREM 1: The is a unique $\theta^*$ such that,
	in any equilibrium of the game with imperfect information,
	the government abandons the currency peg if and only if $\theta \leq \theta^*$. \hspace{\stretch{1}}\parencite[p.~592]{morris1998unique}
\end{quotation}
We will now discuss the issues we find in the derivation of this result.


\section{The Issues}
\label{unc:issues}
By introduction the idea of perceptions of other speculators' perceptions,
the authors state a more realistic model of such a currency attack.
It is thus very surprising that the outcome of their more realistic model
corresponds less to reality, than the more incomplete models do.
After all, in reality, currency attacks do seem to be characterised by multiple equilibria.
This is also explicitly expressed by the authors in their introduction, with the mentioning of \textcite{eichengreen1993unstable} and \textcite{dornbusch1994mexico}.

We believe that this is due to a number of issues in the derivation of the results,
we itentify three issues with this model:
\begin{enumerate}
	\item The unique equilibrium does not apply to the entire `ripe for attack' region.
	\item The measure of risk replaces uncertainty, since previously there were no second-order perceptions.
	\item The model has the unrealistic assumption that distortion would be on a bounded uniform distribution.
\end{enumerate}
In the following subsections, we will describe these problems in more detail.

\subsection{The `Ripe for Attack' Region}
We can summarise the first problem with the model as follows:
\begin{enumerate}
	\item Multiple equilibria exist in the interval $(\underaccent{\bar}{\theta},\bar{\theta})$.
	\item The introduction of quantief risk, removes uncertainty with $2 \epsilon$ of $x$.
	\item Here $u(x,\theta) > 0$ leads to $\pi (x) = 1$.
	\item Authors claim that i.f.f. $x<k$ (the quantified-risk region) speculators will attack. 
	\item This is assumption does not follow from the model.
	\item Outside the quantified-risk area the previously existing uncertainty remains.
	\item Multiple equilibria remain possible here.
\end{enumerate}
Consider the statement of the authors:
\begin{quotation}
	For the next step,
	consider the strategy profile where every speculator attacks the currency if and only if
	the message $x$ is less than some fixed number $k$.
	\hspace{\stretch{1}}\parencite[p.~592]{morris1998unique}
\end{quotation}
The risk profile of the speculator is being redefined here.
In the first iteration of the model,
speculators might decide to the attack the currency without knowing what other investors would do.
In this second iteration,
speculators will only attack if they believe that there is a sufficient number of other speculators doing the same.
Yet in the first model, attacks were possible, in absense of this certainty.
It thus follows that speculators should be willing to attack the currency,
even in outside the quantified-risk region (which now provides certainty).

To illustrate this, consider the following. The risk factor has a magnitude $\epsilon$,
where it is assumed that $\epsilon > 0$.
However, if in stead we state that $\epsilon = 0$, we have eliminated the risk factor.
By eliminating the risk factor, we return to a state of perfect information.
Therefore, solving the second iteration of the model, with $\epsilon = 0$ should give us the same result as the first iteration.
In stead, we find that we end again in a situation of unique equilibria.

As a consequence of this we have the previously mentioned:
\begin{quotation}
	The unique optimal strategy for the government is then to abandon the exchange rate
	only if the observed fraction of deviators, $\alpha$,
	is greater than or equal to the critical mass $a(\theta)$
	in the prevailing state $\theta$.
	\hspace{\stretch{1}}\parencite[p.~591]{morris1998unique}
\end{quotation}
This should say ``...abandon the exchange rate \textit{always} if the observed fraction is great than or equal to...''.
Since quantified-risk region now gives the speculators certainty about the other speculators behavious,
this will now always lead to an sustainable attack.

However, as described above, outside of the new quantied-risk region, the old `ripe for attack region' remains.
In this region of multiple equilibria, currency attacks may be successful, or they may not be.

The condition the governments action is thus a sufficient, but not a necesarry one.


\subsection{Risk and Uncertainty}
The second problem is the most fundamental. The distortion measure is presented as a function that adds uncertainty.
If we differenciate between Knightian uncertainty and risk, then this is not true.

In relation to the first order effect,
the model introduces a quantifiable risk. (which is not the same as uncertainty)

What is more important, is the effect of this risk on the second-order expectations,
that is to say, a speculator's perception of other speculators perceptions.
In the first iteration of the model, there is no second-order expectation,
we are thus in a situation of Knightian uncertainty.
By introducing the quantifiable risk into the second iteration,
\textbf{and making speculators aware of the maximum magnitude of the distortion},
speculators can quantify second-order expectations.

Summierly, the introduction of the distortion, has two effects,
the first-order effect is moving from perfect information to risk,
the second-order effect is moving from uncertainty to risk.
It is this second order effect which leads to the surprising outcome of this model.

\subsection{A Bounded Uniform Distortion}
The third problem is relatively straightforward.
The assumption of a uniform distortion is unorthodox.
We would normally expect that these distortion would take the shape of e.g.~a normal distribution.
The effect of this is extremely relevant.
A uniform distribution is bounded,
which allows the distortion to the quantied with absolute certainty.
An unbounded distribution, such as the normal distribution,
would not allow a speculator to know the limit of other speculators perception,
with absolute certainty.

\section{Conclusion}
\textcite{morris1998unique} presents a model for self-fulfilling currency attacks.
Under perfect information in the first order (and uncertainty in the second order),
this leads to a region of the state of economic fundamentals, where multiple equilibria are possible.
With a introduction of a quantified measure of distortion on the perception of this risk,
the authors claim that the model solves for a unique equilibrium.

We have identified three problems with this model and its outcomes.

Firstly, the new equilibrium, does not cover the entire previously existing multiple equilibria region.
Even though a larger part of the region (w.r.t.~the economic fundamentals) leads to a unique equilibrium,
a part of the old 'ripe for attack' region, remains subject to a situation of multiple equilibria.
It is only through two assumptions which do not follow from the model,
that this region also solves for a unique equilibrium.

Secondly, the most fundamental issue is the distinction between risk and uncertainty.
The first-order effect of the distortion is moving from perfect knowledge to a situation of quantifiable risk.
However, by making speculators aware of the maximum distortion,
they can now form expectations of other speculators expectations.
The second order effect, is thus moving from uncertainty to risk.
This is the driving mechanism behind the new equilibrium in \textbf{part of} the `ripe for attack' region.

Thirdly, the authors introduce a distortion which takes the form of a uniform distribution.
It is only due to the bounded nature of the uniform distribution,
that speculators are able to form bounded quantitative exectatations on other speculators expectations, with absolute certainty.
As mentioned above, it is this absolute certainty on the bounded quantifiable second-order expectations,
that drives the result.

In conclusion, we analyse the model introducted by \textcite{morris1998unique} and find that the results are driven by a second-order effect, as well as certain unjustifiable assumptions.

\vspace{\stretch{1}}

\nocite{taleb2010black}
\printbibliography

\end{refsection}

\begin{refsection}
\chapter{Bitcoin and Remitances: The Potential of `Stupid Phones'}
\label{btc}
In the 2009 whitepaper \parencite{nakamoto2008bitcoin} called ``\citetitle{nakamoto2008bitcoin}'' an anonymous individual or collective referred to a Satoshi Nakamoto (the Japanese equivalent of John Smith) describes an online currency called Bitcoin (BTC). 
This currency -the first of a group called cryptocurrencies- has the following economics features:

\begin{itemize}
\item public transactions
\item anonymous addresses
\item fixed amount of units
\item high granularity (1 Bitcoin = 100,000,000 Satoshi)
\item no or low transaction costs
\end{itemize}

An addition to these economic features, the key feature of Bitcoin is decentralisation.
All transactions are kept in a public ledger called the blockchain. An incentive in the form of transaction fees is provided to stimulate the creation of multiple copies of this ledger.
Conflict resolution, in this system, is done through majority consensus of blockchain copy holders.

There are a number of features which make Bitcoin attractive for remittances. Firstly, the low transaction costs are a huge step up from the current facilitators, which charge high percentages. Secondly, government intervention (especially in the recipient country) is prevented by the anonymity. Thirdly, recipients can choose to keep their remittance in a currency which is not controlled by their government. Fourthly, the granularity ensures very small transactions can be conducted.

Bitcoin initially because popular in advanced economies such as those in Europe and North America.
However, since mid 2013 there has been a large shift towards developing economies,
in which comparative advantages of Bitcoin are more pronounced.
For example, since its inception, Mt.Gox (located in Japan) was the largest Bitcoin exchange (marketplace), however in October 2013, Mt.Gox was overtaken by a Chinese exchange.

Bitcoin transactions are conducted by the electronic `signing' of a transaction.
This is a very simple electronic process.
Currently many smartphone apps are available to this purpose.
However, in the targeted developing economies, many phones are not smartphones, the vast majority using Nokia Operating System (Nokia OS).
 These phone can also run applications, using the Java Mobile App platform. Currently there are no Bitcoin apps available for this platform.

I propose to participate in the development of open-source applications which would enable the users of stupid phones.
We can then exploit the temporal difference when releasing the applications in different languages at different times,
as well the downloads from traceable IP addresses.

\section{Data}
\label{btc:data}
Funds held at Bitcoin exchanges are not held in the owners bitcoin wallet.
Rather, the exchange holds them in their wallet, and the clients have a claim on this wallet.
Much like the way bank clients have a claim to their money, which is in the bank's vault.

Therefore Bitcoin purchases at exchanges do not lead to transaction being recorded in the blockchain.
Since the Bitcoins remain in the Exchange's wallet, the only thing which changes is the claims of the clients.

\begin{itemize}
 \item The blockchain (public ledger)
 \begin{itemize}
 	\item IP address
 	\item Language of transaction notice
 	\item Transaction times
 \end{itemize}
 \item The exchanges
 \begin{itemize}
 	\item Location of exchanges
 	\item Transaction costs for each currency
	\item The order books
	\item Transaction times
 	\item The traded currency
 \end{itemize}
 \item Software
 \begin{itemize}
 	\item Downloads
 	\item Translation contributions (?)
 	\item Connections to server
 \end{itemize}
\end{itemize}

\printbibliography

\end{refsection}

\begin{refsection}
\chapter{End notes}
\label{end-notes}
In the previous chapters I have presented my current research projects.
These projects are in various stages of completions.

I believe that my South Africa paper provides an interesting opportunity to exploit the data from the household survey,
done before and after a highly relevant policy change.
Especially since the methodology and location are the same as in \textcite{duflo2000child, duflo2003grandmothers}.
In addition to this there is now a third data wave which has become available.
This provides us with more data, and mostly the opportunity to use more interesting methodology.
The text of the paper still has to be written. 

% The replication of Surviving Andersonville is already completed in term of the replication in Stata.
% We would like to redo the analysis in R \parencite{R}.
% The relevance of this for replications can be seen from the recent replication \textcite{bell2013questioning} of the original \textcite{rauchhaus2009evaluating} paper.
% The findings of this replication can summarised as:

%\begin{quote}
%``We were easily able to replicate Rauchhaus’ key findings in Stata, but couldn't get it to work in R. It took us a long while to work out why, but the reason turned out to be an error in Stata: Stata was finding a solution when it shouldn’t'have (because of separation in the data). This solution, as we show in the paper, was wrong – and led Rauchhaus’ paper to overestimate the effect of nuclear weapons on conflict by a factor of several million.''~\parencite{bell2013questioning}
%\end{quote}

% We do not expect to find anything similar to Bell and Miller,
% but think that it can nevertheless be interesting to replicate findings using a different software package.
% In R, we can replicate this using the package called `survival', which supports the Kaplan Meier and NPMLE methods.
% In addition to this, the paper needs to be rewritten to emphasise the key finding of our replication.

The discussion of \textcite{morris1998unique} is in a very early stage.
I have looked at the paper in detail, I have also done a brief analysis of my main critique,
namely the conflation of risk (quantifiable) and Knightian uncertainy (unquantifiable).
However, I would like to presenta worked out example which demonstrates how this convolution leads to the main result of the paper.
In addition to this, this example should demonstrate how only a part of the `ripe for attack' region is solved for the unique equilibrium.

Finally, relating to Bitcoin \parencite{nakamoto2008bitcoin}, I think that there is a enormous potention for uncontrolled low-cost remintances using Bitcoin. I would like to analyse the spread of this. I propose to follow the spread of applications for platforms which are popular in developing economies.

At this point, three out of four of the previously discussed research projects are some form of a replication of prior work. As replications they are mostly critiques of others' work, rather than constructive ideas. This is mostly a result of the fact that these projects progress faster. For this reason, these research projects have progressed more than the projects which are primarily based on my own ideas. However, it is my attention to shift the focus of my thesis to include more original work. Here I present two more ideas which are too preliminary to include in the main body of the text, but are based on my own work.

Firstly, I was in charge of setting up a baseline diagnostic into security perceptions in Conakry, Guinea. This diagnostic will be used to analyse the impact of a police reform conducted here. The reform constitutes of the implementation of a model of `police de proximitee'. The baseline consist of 4500 interviews using smartphones in the comune of Conkary as well a large number of interviews in N'Zerekore (east Guinea). Some time after the implementation of the police reform as follow up study will be conducted. I propose to conduct an impact evaluation using the results from the studies in which I have taken and will take part.

Secondly, I have previously studied philosophy of science in my bachelors in Theoretical Philosophy. I would like to write one paper about an issue relating to this. In particular I would like to at the effect of outliers. The effect how these outliers has extensively been discussed in contexts such as finance and international economics \parencite{taleb2010black,   sornette2009dragon},  as well as indiviual behaviour \parencite{kahneman2011thinking}. I would like to see how these principles affect research in development microeconomics.

In this document I have tried to give a description of my research projects. None of these projects are near completion and many are in fact in a very early stage. The purpose of this has been to describe the direction I am taking with my thesis, and to receive feedback on this. As a result of this being work in progress, the articles are still incomplete, this includes lacking the proper attribution of ideas using citations.

\printbibliography
\end{refsection}

\end{document}
