\begin{refsection}

\chapter{Risk and Uncertainty in Unique Equilibria in Currency Attacks}
\label{unc}
\section*{Abstract}
\textcite{morris1998unique} finds that the introduction of a risk factor leads to a unique equilibrium coming to be in their model of currency attacks.
The self-fullfilling currency attack model which they propose thus has the suprising feature,
that with the introduction of perception distorion, the multiple equilibria region, seems to solve to a single equilibrium.
It is our contention that this is not a surpising result, 
it a consequence of the second-order effect, where the risk factor replaces a previous existant, Knightian uncertainty.
Furthermore, the quantified-risk only solves a certain part of the multiple equilibria region.
Lastly, if we replace the somewhat stange assumption of a bounded uniform distribution of risk perceptions,
with the more orthodox, normaly distribution, we cannot derive the same result.

\section{The Idea}
\label{unc:idea}
This paper addresses the issue multiple equilibria in models of currency attacks.
In situations of a currency peg, there is the possibility of a currency attack.
Speculators will short the pegged currency, hoping the government will release the peg.
If a sufficiently large proportion of the market participates in this,
the cost of maintaining the peg for the government becomes too high,
which will lead to the government releasing the peg.
Currency attacks thus have a self-fulling nature,
which leads to a situation of multiple equilibria,
as described in \textcite{obstfeld1986rational,obstfeld1995logic,obstfeld1996models}.

In \textcite{morris1998unique} the authors describe such a model.
This model is characterised by a situation where there is a stable, unstable,
and a `ripe for attack' region (based on the economic fundamentals.
The authors expand on the standard model by introducting second-order expectations.
Hereby speculators do not only look at the economics fundamentals of a currency,
but also at other speculators' perceptions of these fundamentals.

The authors proceed by introducing a measure of risk around perceptions of the fundamentals by investors.
Investors do not know the actual state of the fundamentals, but rather a distored form of it.
As a result, the investors' perception of other investors' perceptions are even more distorted.
Paradoxically, the leads to model being solvable to a unique equilibrium.

\section{The Model}
\label{unc:model}
I will give a brief description of the model as it is defined, and some it features.
There is a state of economic fundamentals $\theta$ which is distributed as $\theta \sim U[0,1]$.
The exchange rate is the hypothetical situation of no government intervention is a function only of these fundamental ($f(\theta)$),
it is assumed that $\frac{\partial f}{\partial \theta} > 0$.
The currency is pegged at a level larger than that derived from the fundamentals ($e* \geq f(\theta)$).
Speculators can short the currency at a cost $t$, their payoff is described by:
\begin{equation}\label{unc:specPO}
e^* - f(\theta) - t
\end{equation}
The government derives a positive value $\nu > 0$ from defending the peg.
The cost of the peg is determined by the economic fundamentals and the proportion of the speculators ($\alpha$) that attack the currency.
The payoff the government is thus:
\begin{equation}\label{unc:govPO}
\nu - c(\alpha, \theta)
\end{equation}
Lastly, the following, assumptions are imposed: $c(0,0) > \nu$ (with worst fundamtentals the peg always has to be released),
$c(1,1) > \nu$ (with all speculators attacking, the peg always has to be released),
$e^* - f(1) < t$ (with best fundamentals, it is inopportune for speculators to attack).

\subsection{The possible outcomes}
When we set \autoref{unc:specPO} and \autoref{unc:govPO} equal to zero,
we derive the two turning points in the model.
We define $\underaccent{\bar}{\theta}$ which solves $c(0, \theta) = \nu$.
Also, $\bar{\theta}$ solves $f(\theta) = e^* - t$.

Using the turning points described above, we can define three possible outcome intervals:
\begin{enumerate}
	\item $[0, \underaccent{\bar}{\theta}]$, the cost is always too high, this is the unstable region.
	\item $[\underaccent{\bar}{\theta}, \bar{\theta}]$, if enough speculators attack, the cost of defending the peg becomes too high,
	this is the `ripe for attack' region.
	\item $[\bar{\theta}, 1]$, the cost of shorting the currency will always outway the possible gains,
	this is the stable region.
\end{enumerate}
The two corner intervals have unique equilibria, however, the middle interval, does not. This is thus the multiple equilibria region.


\section{The introduction of risk (replacing uncertainty)}
The authors expand the model by introducing a measure of risk for fundamentals:
\begin{equation}
x \sim U[\theta - \epsilon, \theta + \epsilon]
\end{equation}
Where is the $x$ is the state of the fundamentas \textit{observed} by the speculators.
A speculators observation can thus \textit{at most} deviate from the true value of the fundamentals by $\epsilon$.
The speculators are also aware of the nature of the distortion.
This means that they are also aware of the fact that other speculators' perceived value of the fundatmentals can at most,
deviate from their own with a magnitude of $2\epsilon$.

\subsection{The outcomes with risk}
\label{unc:results}
Having established the new model, the authors state:
\begin{quotation}
	The unique optimal strategy for the government is then to abandon the exchange rate only if the observed fraction of deviators, $\alpha$, is greater than or equal to the critical mass $a(\theta)$ in the prevailing state $\theta$.\hspace{\stretch{1}}\parencite[p.~591]{morris1998unique}
\end{quotation}
By solving out this government `optimal' strategy,
the speculators can ensure themselves of a certain abandonment of the peg,
for a sufficiently low observed $x$.
The authors proceed to derivde from this the main result of the paper, which is:
\begin{quotation}
	THEOREM 1: The is a unique $\theta *$ such that,
	in any equilibrium of the game with imperfect information,
	the government abandons the currency peg if and only if $\theta \leq \theta *$. \hspace{\stretch{1}}\parencite[p.~592]{morris1998unique}
\end{quotation}


\section{The Issues}
\label{unc:issues}
By introduction the idea of perceptions of other speculators' perceptions,
the authors give a more complete picture of such a currency attack.
It is thus very surprising that the outcome of their more realistic model,
corresponds less to reality and the more incomplete models.
That is to say, the nature of currency attacks does seem to be characterised by multiple equilibria.
This explicitly expressed by the authors in their introduction, with the mentioning of \textcite{eichengreen1993unstable} and \textcite{dornbusch1994mexico}.
It is therefore quite surprising that the more realistic leads to a the unrealistic situation of a unique equilibrium,
whereas the less intricate model gives us the more realistic multiple equilibria.

We itentify three issues with this model:
\begin{enumerate}
	\item The unique equilibrium does not apply to the entire `ripe for attack' region.
	\item The measure of risk replaces uncertainty, since previously there were no second-order perceptions.
	\item The model has the unrealistic assumption that distortion would be on a bounded uniform distribution.
\end{enumerate}
In the following subsections, we will describe these problems in more detail.

\subsection{The `Ripe for Attack' Region}
We can summarise the first problem with the model as follows:
\begin{enumerate}
	\item Multiple equilibria exist in the interval $(\underaccent{\bar}{\theta},\bar{\theta})$.
	\item The introduction of quantief risk, removes uncertainty with $2 \epsilon$ of $x$.
	\item Here $u(x,\theta) > 0$ leads to $\pi (x) = 1$.
	\item Authors claim that i.f.f. $x<k$ (the quantified-risk region) speculators will attack. 
	\item This is assumption does not follow from the model.
	\item Outside the quantified-risk area the previously existing uncertainty remains.
	\item Multiple equilibria remain possible here.
\end{enumerate}
Consider the statement of the authors:
\begin{quotation}
	For the next step,
	consider the strategy profile where every speculator attacks the currency if and only if
	the message $x$ is less than some fixed number $k$.
	\hspace{\stretch{1}}\parencite[p.~592]{morris1998unique}
\end{quotation}
The risk profile of the speculator is being redefined here.
In the first iteration of the model,
speculators might decide to the attack the currency without knowing what other investors would do.
In this second iteration,
speculators will only attack if they believe that there is a sufficient number of other speculators doing the same.
Yet in the first model, attacks were possible, in absense of this certainty.
It thus follows that speculators should be willing to attack the currency,
even in outside the quantified-risk region (which now provides certainty).

Secondly, the previously mentioned:
\begin{quotation}
	The unique optimal strategy for the government is then to abandon the exchange rate
	only if the observed fraction of deviators, $\alpha$,
	is greater than or equal to the critical mass $a(\theta)$
	in the prevailing state $\theta$.
	\hspace{\stretch{1}}\parencite[p.~591]{morris1998unique}
\end{quotation}
This should say ``...abandon the exchange rate \textit{always} if the observed fraction is great than or equal to...''.
Since quantified-risk region now gives the speculators certainty about the other speculators behavious,
this will now always lead to an sustainable attack.

However, as described above, outside of the new quantied-risk region, the old `ripe for attack region' remains.
In this region of multiple equilibria, currency attacks may be successful, or they may not be.

The condition the governments action is thus a sufficient, but not a necesarry one.


\subsection{Risk and Uncertainty}
The second problem is the most fundamental. The distortion measure is presented as a function that adds uncertainty.
If we differenciate between Knightian uncertainty and risk, then this is not true.

In relation to the first order effect,
the model introduces a quantifiable risk. (which is not the same as uncertainty)

What is more important, is the effect of this risk on the second-order expectations,
that is to say, a speculator's perception of other speculators perceptions.
In the first iteration of the model, there is no second-order expectation,
we are thus in a situation of Knightian uncertainty.
By introducing the quantifiable risk into the second iteration,
\textbf{and making speculators aware of the maximum magnitude of the distortion},
speculators can quantify second-order expectations.

Summierly, the introduction of the distortion, has two effects,
the first-order effect is moving from perfect information to risk,
the second-order effect is moving from uncertainty to risk.
It is this second order effect which leads to the surprising outcome of this model.

\subsection{A Bounded Uniform Distortion}
The third problem is relatively straightforward.
The assumption of a uniform distortion is unorthodox.
We would normally expect that these distortion would take the shape of e.g.~a normal distribution.
The effect of this is extremely relevant.
A uniform distribution is bounded,
which allows the distortion to the quantied with absolute certainty.
An unbounded distribution, such as the normal distribution,
would not allow a speculator to know the limit of other speculators perception,
with absolute certainty.

the equilibrium is most pronounced when the risk factor ($\epsilon$) is minimised (i.e.~set equal to zero),
as the authors state themselves. % FIND THIS CLAIM!!!!

\section{Conclusion}
\textcite{morris1998unique} presents a model for self-fulfilling currency attacks.
Under perfect information in the first order (and uncertainty in the second order),
this leads to a region of the state of economic fundamentals, where multiple equilibria are possible.
With a introduction of a quantified measure of distortion on the perception of this risk,
the authors claim that the model solves for a unique equilibrium.

We have identified three problems with this model and its outcomes.

Firstly, the new equilibrium, does not cover the entire previously existing multiple equilibria region.
Even though a larger part of the region (w.r.t.~the economic fundamentals) leads to a unique equilibrium,
a part of the old 'ripe for attack' region, remains subject to a situation of multiple equilibria.
It is only through two assumptions which do not follow from the model,
that this region also solves for a unique equilibrium.

Secondly, the most fundamental issue is the distinction between risk and uncertainty.
The first-order effect of the distortion is moving from perfect knowledge to a situation of quantifiable risk.
However, by making speculators aware of the maximum distortion,
they can now form expectations of other speculators expectations.
The second order effect, is thus moving from uncertainty to risk.
This is the driving mechanism behind the new equilibrium in \textbf{part of} the `ripe for attack' region.

Thirdly, the authors introduce a distortion which takes the form of a uniform distribution.
It is only due to the bounded nature of the uniform distribution,
that speculators are able to form bounded quantitative exectatations on other speculators expectations, with absolute certainty.
As mentioned above, it is this absolute certainty on the bounded quantifiable second-order expectations,
that drives the result.

In conclusion, we analyse the model introducted by \textcite{morris1998unique} and find that the results are driven by a second-order effect, as well as certain unjustifiable assumptions.

\nocite{taleb2010black}
\printbibliography
\end{refsection}