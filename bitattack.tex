\begin{refsection}
\chapter{Bitcoin Attacks: Or How Cryptocurrencies Gain Momentum}
\label{bitatt}
\vspace{\stretch{1}}
\section*{Abstract}
In this chapter I set out the idea of employing the models used for currency attacks,
to analyse how cryptocurrencies gain momentum.
Additionaly, I discuss how inflation in the cryptocurrency economy is possible.
\pagebreak

\section{Introduction}
As discussed in \autoref{bitremit}, there has been a new phenonemon in international finance called cryptocurrencies.
The first of these was the cryptocurrency called Bitcoin \parencite{nakamoto2008bitcoin}.
However, soon after Bitcoin started gaining momentum, alternative crypotcurrencies where being designed.
All of these cryptocurrencies are based to a large extend on the Bitcoin protocol,
but often introduce small technological innovations.

The most known of all of these alternative crypocurrencies is Litecoin. Litecoin introduces a number of small technological innovations,
the most import of these is the change that the process of `mining' coins is 4 times faster,
this has two consequences, transactions are approved faster than in the bitcoin protocol.
It also means that the technological limit for the total number of Litecoins ever available is 4x times higher.
For Bitcoin the technical limit is around 21 million Bitcoin, which is expected to be reached in the year 2140.
For Litecoin, the limit is around 84 million bitcoin, which is also expected to be reached around this time.

The key thing to note here is the perspective of the new cryptocurrecy adopt.
Even though the value of Bitcoin is much higher than that of Litecoin, it does not have more to offer.
The lower valued new cryptocurrency is technologically superior and thus more attractive.
Most new cryptocurrencies do not gain momentum and their value is only trivialy more than zero.
However, some currencies have gain momentum,
the previously mentioned litecoin introduces a technological innovation and build it base on that.
However, recently a new cryptocurrency has arisen and become quite popular.
This cryptocurrency is called `Dogecoin' and besides it name, it does not introduce any novelty.
Dogecoin is a reference to an internet phenomenon called `Doge Meme'.
Despite the fact that `Dogecoin' does not have a real reason for being considered valuable
(in the sense that crypotocurrencies are considered valuable),
however, by utilising a previously existing internet phenomenon, and it's fanbase, this currencies has become valuable.
We thus observe that from the perspective of economic and technological fundamentals,
there is a situation of multiple equilibria.

\section{Positive Currency Attacks}
My idea is to adapt standard models of currency attacks such as in 
\textcite{obstfeld1986rational,obstfeld1995logic,obstfeld1996models}
to model the gaining of long (v.s.~currency attacks' short) momentum for cryptocurrencies.
Initially we would model how a currency can gain sufficient momentum to establish itself as a stably above zero.

\subsection{A Multi-currency Model}
This could later be expended to a model with several cryptocurrencies,
competing for the investment of the same new cryptocurrency adopt,
who is converting his/her fiat currency into one of several cryptocurrencies.

\section{Multi-cryptocurrency Economy and Inflation}
Finally, an extension of such a multi-cryptocurrency model can provide a valuable insight into the issue of inflation.

On of the key selling points of the cryptocurrencies, and equally one of the key critisism of many economics
(e.g.~Paul Krugman), is that there is no inflation in a crypocurrency.
The crypocurrencies are designed specifically so that inflation eventually becomes impossible (once all coines are mined).

However, it is key here to understand the economy as a multi-cryptocurrency object.
Since creating a currency can be done at negligable cost,
and the only value of the cryptocurrency lies in its utility as a mechanism for transactions,
more cryptocurrencies will come into existance.
Especially since it is very simple for a merchant to accept several cryptocurrencies as a form of payment.
The effect of this will be that inflation in each cryptocurrency will remain as defined,
however, the cryptocurrency economy as a whole, will be able to experience inflation.

Additionaly, this partly addresses one of social critiques of Bitcoin,
namely the issue of bitcoin mining being excessively wastful.
Bitcoin mining does not produce anything of any social value (not unlike gold mining),
the higher value of bitcoin, the more computing power (and electricity) will be diverted to this.
However, in a multi-currency model, it will be more effective to mine alternative cryptocurrencies,
as these can initially be mined with relative ease.


\begin{figure}
\caption{A Doge Meme}
\label{doge}
%\includegraphics[scale=0.8]{doge.jpg}
\end{figure}

\printbibliography
\end{refsection}