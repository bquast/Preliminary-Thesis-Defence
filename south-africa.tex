\begin{refsection}
\chapter{Pensions and Child Growth in South Africa}
\label{sa}
\vspace{\stretch{1}}
\section*{Abstract}
In this paper we look at variation in the health of young children driven by the gender of the household income recipient.
We do this by comparing z-scores of anthropometrics of South-African children living in the same household as state pension recipients.

This paper exploits the lowering of the state-pension eligibility-age of men, 
to the same age as women (60, previously 65).
This takes place between two waves in the South-African National Income Dynamics Survey.
This enables us to perform a Difference-in-Difference estimation on the panel data set.

Our finding is that policy change had a negative effect on long-term growth metrics of young children
and the general mens pension income had a negative effect on young children's BMI.
\pagebreak

\section{Introduction}
\label{sa:intro}
This paper looks at the effect of the gender of pension recipients on the growth of children in the same household.
The study is based on South-African data and the approach is very similar to \textcite{duflo2000child,duflo2003grandmothers}.
The difference from international standards \parencite[WHO Child Growth Standards]{who2006child} for anthropometrics are computed as z-scores.
Using these standardised metrics, we compare children living in household with pension recipients of different gender.

This study deviates from the Duflo study in several ways.
The core contribution of this paper is the analyis of exogenous change in men's pension eligibility age,
which is the main explanatum.
The pension eligibility age for men was lowered from 65 to 60 between mid 2009 and 1-1-2011.
This brought the pension eligibility age for men at par with women.
We employ a Difference-in-Difference (or Fixed Effects) analysis of this change.
Under then assumptions of the Difference-in-Difference model,
this enables us to make a causal inference on the policy variable.

The other deviations are of a more practical nature.
Firstly, the data from the \textcite{saldru2008nids,saldru2012nids} surveys contains actual information on income,
including pension recipient status, whereas Duflo uses age as a proxy for recipient status.
Secondly, another minor deviation is the usage of \textcite[WHO Child Growth Standards]{who2006child},
in stead of \textcite[CDC Growth Charts: United States]{nchs2000cdc}, since these have superseded the CDC charts.
As long as all observations are held against the same standards, this should not be of any consequence.

The impertus for this paper lies in the optimal design of cash transfer schemes such as CCTs and UCTs.
The lack Paretto optimal allocation of resources within households as discussed in i.a.
\textcite{udry1995gender}, \textcite{udry1996gender}, and \textcite{duflo2004intrahousehold},
indicates the necesity of optimal design in such schemes.
Based on this lack of Paretto optimal allocation,
we have to reject the idea of households acting as a unit in an economic sense.
For the design of cash transfer schemes it is therefore necessary to determine the prefered recipient within the household.

The South African pension system is an interesting object of study because of its eligibility criteria.
The primary criterium is the age of the recipient.
In adition to this there is a maximum income threshold.
Outside of this, there are very few criteria.
The reletive general applicablility of the program makes that there are few selection bias issues when studying this.

The anthropometrics are useful for computing z-scores.
These z-scores are considered a good representation of short-term or long-term health issues, respectively.
This relation is especially well observed for children between 6 and 60 months old.
We therefore stay with the best practice and only include those observations in our analysis.

% HISTORY OF APARTHEID AND CHANGE IN PENSION SCHEME

% ESTIMATION

Our main finding is that we find a negative effect of the policy change on the age-based growth metrics on children (HAZ and WAZ).
In the height-based metrics we find a negative effect of the state pension income of men on the body mass index (though not on the WHZ).
Our results seems to indicate that the policy change had a negative impact on the long-term growth of children.
Furthermore, we see a negative effect of the mens pension income on the BMI of children.
These results provide support for the theory that exogenous incomes in a poverty relief context,
such as CCTs and UCTs are best transfered to women in the households.

\section{Data}
\label{sa:data}
% DESCRIPTIVE STATISTICS OF THE DATASET AND POSSIBLY THE GROWTH STANDARDS
In this paper we use data from two sources.
The first is the South African National Income Dynamics Survey \parencite[NIDS][]{saldru2008nids, saldru2012nids, saldru2013nids} and the second is the World Health Organization's Child Growth Standards \parencite[WHO]{who2006child}.

\subsection{South Africa: National Income Dynamics Survey}
The main source of data is the National Income Dynamics Survey of South Africa \parencite{saldru2008nids, saldru2012nids, saldru2013nids}.
Like the 1993 survey used by \textcite{duflo2000child, duflo2003grandmothers}, this survey is conducted in cooperation with the World Bank.
Unlike the 1993 survey, this survey does not use a random selection household,
rather it collects data on a representative set of appproximately 10,000 South-African households over time.
Currently three `waves' of data are available, these waves date from 2008, 2012, and 2013 respectively.

The primary information types we use are:
\begin{itemize}
  \item child anthropometrics,
  \item child age (in days)
  \item child gender
  \item adult pension recipient status
  \item adult gender
\end{itemize}
In addition to these variables of interest, we include a number of covariates in the analysis. These are
\begin{itemize}
  \item Household income
  \item Parent's education
\end{itemize}

For adults several variables measure the different amounts and sources of income.
Among those, a variable if the adult receives a state pension, and if so, how much.
This is a numeric variable, the values of which lie very close together.

Children's anthropometrics are taken, these are length/height, weight, and waist.
Using these anthropometics and WHO growth standards, z-scores are calculated.


\subsection{WHO: Child Growth Standards}
In 2006 the WHO published its standards for child growth \parencite{who2006child}.
These standards measure the difference between a child's anthropometrics
standardised against an ideal score.

Z-score anthropometrics are used since they are considered to be a good representation of a child's health,
and by extension, the household in which they grow up.
These ideal scores are based on a sample of children from different ethnic populations,
in households which observed a healthy lifestyle.
Any health issues, such as malnutrition or disease will affect these metrics,
by causing the child to be shorter or lighter.
However, it is impossible to distinquish between the different causes of an observed slowed growth.

We stay with the best practice of using only metrics for children between the ages of 6 months and 60 months.

In general, can distinquish between two types of anthropometric z-scores, the age-based z-scores and the height-based z-scores.
Whereby `based' refers to the reference point at which anthropometrics are standardised.

\subsubsection{Age-Based Z-scores}
The Age-Based Z-scores (ABZs) are consistituted by the Height-for-Age Z-score (HAZ) and the Weight-for-Age Z-score (WAZ).
Since these metrics are age-based, they provide information about all past growth issues.
Any past issues such a malnutrition and disease will have impaired growth,
and these effects will still be captured by today's height.
This also applies to the WAZ, as standard weight is a function of the height,
which is in turn a function of the age.

The ABZs are constructed on a weekly basis up to the age of 60 months, and on a monthly basis thereafer.


\subsubsection{Height-Based Z-scores}
The Height-Based Z-scores (HBZs) are the Weight-for-Height Z-score (WHZ)
and Body Mass Index Z-score (BMIZ).
Where the BMI (or Quetelet) is a transformed version of the WHZ, which has a quadratic height effect.
The equation for BMI is:
\[
\textbf{BMI} = \frac{\textbf{weight(kg)}}{\textbf{height(m)}^2}
\]

These scores compare children with others of the same height, irrispective of their age.
As a results we only observe the relatively short-term effect of weight.
The height-based metrics thus provide is with a short-term insight.

The HBZs are available on a semi-centimeter level throughout all heights.

\subsection{Data Structure}
The NIDS uses a file and data structure which is ill suied for panel data analysis.
We therefore transform the data to a format which is more conducive to our analysis.
In doing so, we try to stay as close as possible to the`Tidy data' structure, as described in \textcite{wickham2014tidy}.
This is easiest using the R package `Reshape2' by the same author \parencite[Reshape2 implementation]{wickham2007reshaping}.

\section{Research Design}
This study focuses on a policy change in the South-African state pension system.
Until mid 2009, men became eligible for pension at the age of 65.
Between mid 2009 and January 1st 2011, this was gradually lowered to 60.
The South-African National Income Dynamics Survey is a full-panel dataset,
which contains information on household from before and after this policy change.
We study the effect of the policy change,
as well as the general effect on the pension system,
on the health of children in the same household.
The research setup in discussed in further detail below.


\subsection{Identification Strategy}
\label{sa:identification}
The identification strategy in this paper is based on a policy change in the pension eligibility age for men,
which was introduced between mid 2009 and January 1st 2011.
This policy change thus fall between waves 1 and 2 (2008 and 2012 respectively) of the NIDS data sets.

Before this policy change, the eligibility age for men was 65 years old.
Post the policy change, the eligibility age is 60 years old,
which bring it at par with the pension eligibility age for women.

We operationalise this natural experiment, by constructing a policy dummy.
This policy dummy is called \textbf{elig.men.60}, and takes the value \textbf{1}
for data after the policy change (i.e.~waves 2 \& 3), and the value \textbf{0} otherwise (i.e.~wave 1).

\subsection{Estimator}
In order to fully exploit the available data and the policy change,
we employ a `Difference-in-Differences' estimator (DiD).
This estimator operationalised by using the fixed-effects (within) estimator, with a time-effect\footnote{The time effect estimated here is symmetric to the individual effect.
We employ the term `time effect', since it is a more meaningful description of the policy change.}.

We perform the estimations using the R package `PLM' \parencite[see][]{croissant2008panel}.
It is worth noting  that questions have been raised about the Difference-in-Differences estimator being employed in certain situations,
for example by (ironically) \textcite{bertrand2004much}.

\subsubsection{Formal DiD Specification}
We can now formally specify our estimation as:
\begin{equation}
y_{it} = \gamma_i + \lambda_t + \delta T_{it} + P^f_{it} + P^m_{it} + X_{it} + \epsilon_{it}
\end{equation}
The outcome variable is $y_{it}$, this outcome variable takes the form of the of the z-scores,
such as HAZ or WAZ.
Where $t$ denotes the time and $i$ the individual.
The individual and time fixes effects are denoted by $\gamma_i$ and $\lambda_t$ respectively.
Dummies for living in a household with a female or a male pension recipient are included as $P^f_{it}$ and $P^m_{it}$ respectively.
The dummy variable $T_{it}$ denoted the treatment status.
Lastly, $\epsilon_{it}$ is the error term, which is assumed to be distributed as:
\[
\epsilon_{it} \sim N(0,\sigma)
\]
As we have only one time period before the treatment goes into effect,
we cannot establish a common trend.
The assumption here made is thus that the effects of $P^f_{it}$ and $P^m_{it}$ are level over time.

\section{Results}
In \autoref{sa:hazwaz} we present our estimation results for the age-based z-scores,
and in \autoref{sa:whzbmiz} we present our estimation results for the height-based z-scores.

% LaTeX table created manually
% Fri Dec 27 23:22:00 2013
\begin{table}[!ht]
\centering
\caption{Age-Based Z-scores}
\label{sa:hazwaz}
\begin{tabular}{l|rrr|rrr}
% \hline
& \multicolumn{3}{c}{Height for Age Z-score} & \multicolumn{3}{c}{Weight for Age Z-score}\\
\hline
specification & 1 & 2 & 3 & 1 & 2 & 3\\
\hline
w\_spen\_m & 0.2366 & *0.8228 & 0.7908 & 0.2366 & 0.2981 & 0.4780 \\
w\_spen\_w & -0.2331 & 0.1053 & 0.1072 & -0.2331 & -0.3112 & -0.3280 \\
elig.men.60 & & **-0.3419 & **-0.3465 & & ***-0.3475 & **-0.3243 \\
w\_spen\_m1:elig.men.60 & & & 0.0446 & & & -0.2545 \\
% \hline
w\_h\_tinc & -0.0000 & -0.0000 & -0.0000 & -0.0000 & -0.0000 & -0.0000 \\
\end{tabular}
\end{table}

% LaTeX table created manually
% Fri Dec 27 23:53 2013
\begin{table}[!ht]
\centering
\caption{Height-Based Z-scores}
\label{sa:whzbmiz}
\begin{tabular}{l|rrr|rrr}
& \multicolumn{3}{c}{Weight for Height Z-score} & \multicolumn{3}{c}{Body Mass Index Z-score}\\
\hline
specification & 1 & 2 & 3 & 1 & 2 & 3 \\
\hline
w\_spen\_m &  -0.3532 & -0.3210 & -0.4303 & *-0.8058 & *-0.7905 & *-1.0226 \\
w\_spen\_w & 0.0655 & 0.0371 & 0.0478 & -0.1592 & -0.1956 & -0.1742 \\
elig.men.60 & & -0.1417 & -0.1574 & & -0.1674 & -0.2049 \\
w\_spen\_m1:elig.men.60 & & & 0.1484 & & & 0.3407 \\
w\_h\_tinc & -0.0000 & -0.0000 & -0.0000 & -0.0000 & 0.0000 & 0.0000
\end{tabular}
\end{table}



\subsection{Robustness Checks}
% HAUSMAN TEST FOR FE / RE

\section{Conclusion}

\printbibliography
\end{refsection}