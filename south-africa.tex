\begin{refsection}
\chapter{Pensions and Child Growth in South Africa}
\label{sa}
\section*{Abstract}
In this paper I look at differences in child malnutrition levels depending on the gender of the recipient of income.

I do this by comparing z-scores of anthropometrics of South African children living in the same household as state pension recipients.
This paper exploits the fact that the data set consists of three surveys,
one of which was done before the lowering of the pension eligibility age for men to the same age as women,
and two after.

The main preliminary finding is that the household effect was only significant in 2012.

\section{Introduction}
\label{sa:intro}
This paper looks at the effect of the gender of pension recipients on the growth of children in their household in South Africa.
The approach is very similar to \textcite{duflo2000child,duflo2003grandmothers}.
The difference from international standards \parencite[WHO Child Growth Standards]{who2006child} for weight-for-height and length-for-age are computed as z-scores.
These are then compared for different pension recipient status.

The anthropometrics are useful for computing z-scores. These z-scores are considered a good representation of short-term or long-term malnutrition respectively, especially for children between 6 and 60 months old.

The South African pension is a useful variable to measure income because of its criteria.
Besides a maximum level of income, the only criterium is the age of a person.
Because of this the status as a recipient is quite exogenous and there are few selection bias problems.
The problematic difference between the eligibility age of men and women was eliminated between the two surveys which creates an interesting natural experiment.

This study deviates from the Duflo study in several ways.
First of all, the data from the \textcite{saldru2008nids,saldru2012nids} surveys, contains actual information on income,
including pension recipient status.
Whereas Duflo uses age as a proxy for recipient status.
Secondly, between mid 2008 and the end of 2009 the pension age for men was gradualy lowered to sixty, which is at par with women.
Another deviation is the usage of \textcite[WHO Child Growth Standards]{who2006child} in stead of \textcite[CDC Growth Charts: United States]{nchs2000cdc}.
Since these have superseded the CDC charts, however this should not be of any consequence.

The main preliminary result is that the household effect (i.e. having one or more recipients in the household),
is very insignificant in the 2008 estimate.
But in the 2012 estimate it is very significant.

\section{Data}
\label{sa:data}
In this paper I use data from two sources.
The first is the South African National Income Dynamics Survey \parencite[NIDS]{saldru2008nids, saldru2012nids, saldru2013nids} and the second is the World Health Organization's Child Growth Standards \parencite[WHO]{who2006child}.

The main source of data is the NIDS.
This survey collects data on a representative set of appproximately 10,000 South-African households.
The primary information types I use are, the child anthropometrics, the age and gender of household members, and the status as state pension recipients.

For adults several variables measure the different amounts and sources of income.
Among those, a variable if the adult receives a state pension, and if so, how much.
This is a numeric variable, the values of which lie very close together.
For simplicity I have temporarily used this variable as a dummy.

Children's anthropometrics are taken, these are length/height, weight, and waist.
Using these anthropometics and WHO growth standards, z-scores have been calculated.
Unfortunately wave 2 (2012) accidentally omitted the z-scores, so that these cannot be evaluated until an updated version is published.
However, I have computed the length-for-age z-scores manually.

In 2006 the WHO published its standards for child growth \parencite{who2006child}.
These standards are based on the scores of children from different ethnic populations in households which observed a healthy lifestyle.
The standards provide the means and standards deviations used. 
These are on a monthly basis for height-for-age, and on a semi-centimeter level for weight-for-height scores.

\section{Identification Strategy
\label{sa:identification}}
The policy change, a natural experiment

the ATET, etc.

\subsection{The estimation using OLS}
Our first estimation method uses OLS.

\subsection{Difference in Differences}
The estimation using diff-in-diff, its equivalence to fixed-effects, and how it is different when using T=3.

\section{Obsolete: Analysis}
\label{sa:analysis}
I perform three types of analysis.
The first analysis uses the Duflo method, whereby I compute the z-scores (using WHO in stead of CDC, this is of no consequence)
and then compare based on living in the household with an of-pension-eligible adult.
Secondly, I use my computed z-scores and the data set provided variable which describes the actual pension received (or lack thereof).
Lastly, I use this state pension variable to evaluate the z-scores provided in the dataset.

The first type of analysis is almost identical to Duflo.
Using an internation set of standard anthropometrics for healthy children between 6 and 60 months old, I standarize observed anthropometrics.
I then construct a number of dummies of different ages and gender of the pension recipients.

In the second analysis, I use the same z-scores I calculated for the Duflo method.
However, here I use the state pension recipient variable provided by the NIDS data set.
For simplicity I change the numerical variable to a dummy variable (as the benefits are virtually identical this is without much loss of information).

This method uses the pre calculated z-scores from the data set and the state pension dummy.
Unfortunately the 2012 z-scores are missing except for height-for-age.
I therefore use only the height-for-age z-scores for the 2012 estimates.

In all of these methods the z-scores can be compared on a number of properties.
Firstly, the status of living in the same household as a state pension recipient or not is the first property to compare.
Secondly, when living with a recipient,the effects of male and female recipients can be compared.
Thirdly, the effects based on the gender of the child can be compared.

\printbibliography
\end{refsection}