\begin{refsection}
\chapter{Pensions and Child Growth in South Africa}
\label{sa}
\section*{Abstract}
In this paper I look at differences in child malnutrition levels depending on the gender of the recipient of income.

I do this by comparing z-scores of anthropometrics of South African children living in the same household as state pension recipients.
This paper exploits the fact that the data set consists of three surveys,
one of which was done before the lowering of the pension eligibility age for men to the same age as women,
and two after.

The main preliminary finding is that the household effect was only significant in 2012.

\section{Introduction}
\label{sa:intro}
This paper looks at the effect of the gender of pension recipients on the growth of children in their household in South Africa.
The approach is very similar to \textcite{duflo2000child,duflo2003grandmothers}.
The difference from international standards \parencite[WHO Child Growth Standards]{who2006child} for weight-for-height and length-for-age are computed as z-scores.
These are then compared for different pension recipient status.

The anthropometrics are useful for computing z-scores. These z-scores are considered a good representation of short-term or long-term malnutrition respectively, especially for children between 6 and 60 months old.

The South African pension is a useful variable to measure income because of its criteria.
Besides a maximum level of income, the only criterium is the age of a person.
Because of this the status as a recipient is quite exogenous and there are few selection bias problems.
The problematic difference between the eligibility age of men and women was eliminated between the two surveys which creates an interesting natural experiment.

This study deviates from the Duflo study in several ways.
% MAIN DEVIATION, PANEL NATURAL EXPERIMENT
First of all, the data from the \textcite{saldru2008nids,saldru2012nids} surveys, contains actual information on income,
including pension recipient status.
Whereas Duflo uses age as a proxy for recipient status.
Secondly, between mid 2008 and the end of 2009 the pension age for men was gradualy lowered to sixty, which is at par with women.
Another deviation is the usage of \textcite[WHO Child Growth Standards]{who2006child} in stead of \textcite[CDC Growth Charts: United States]{nchs2000cdc}.
Since these have superseded the CDC charts, however this should not be of any consequence.

% THE IMPERTUS FOR THE PAPER IS THE GENDER IN CCTs and OPTIMAL DESIGN
% RELATES TO THE ISSUE OF HOUSEHOLD AS A UNIT
% THEY ARE NO LONGER THAT RARE IN DEVELOPING COUNTRIES ???

% HISTORY OF APARTHEID AND CHANGE IN PENSION SCHEME

% STATE THE MAIN RESULT

\section{Data}
\label{sa:data}
% DESCRIPTIVE STATISTICS OF THE DATASET AND POSSIBLY THE GROWTH STANDARDS
In this paper I use data from two sources.
The first is the South African National Income Dynamics Survey \parencite[NIDS][]{saldru2008nids, saldru2012nids, saldru2013nids} and the second is the World Health Organization's Child Growth Standards \parencite[WHO]{who2006child}.

\subsection{South Africa: National Income Dynamics Survey}
The main source of data is the National Income Dynamics Survey of South Africa \parencite{saldru2008nids, saldru2012nids, saldru2013nids}.
This survey collects data on a representative set of appproximately 10,000 South-African households over time.
The primary information types I use are:
\begin{itemize}
  \item child anthropometrics,
  \item child age (in days)
  \item adult pension recipient status
  \item adult pension recipient gender
\end{itemize}

For adults several variables measure the different amounts and sources of income.
Among those, a variable if the adult receives a state pension, and if so, how much.
This is a numeric variable, the values of which lie very close together.
For simplicity I have temporarily used this variable as a dummy.

Children's anthropometrics are taken, these are length/height, weight, and waist.
Using these anthropometics and WHO growth standards, z-scores have been calculated.

\subsection{WHO: Child Growth Standards}
In 2006 the WHO published its standards for child growth \parencite{who2006child}.
These standards measure the difference between a child's anthropometrics
standardised against an ideal score.

Z-score anthropometrics are used since they are considered to be a good representation of a child's health,
and by extension, the household in which they grow up.
These ideal scores are based on a sample of children from different ethnic populations,
in households which observed a healthy lifestyle.
Any health issues, such as malnutrition or disease will affect these metrics,
by causing the child to be shorter or lighter.
However, it is impossible to distinquish between the causes of an observed slow growth.

We stay with the best practice of using only metrics for children between the ages of 6 months and 60 months.

In general, can distinquish between two types of anthropometric z-scores, the age-based z-scores and the height-based z-scores.
Whereby `based' refers to the reference point at which anthropometrics are standardised.

\subsubsection{Age-Based Z-scores}
The age-based z-scores are the Height-for-Age Z-score (HAZ) and the Weight-for-Age Z-score (WAZ).
Since these metrics are age-based, they provide information about all past growth issues,
any past issues such a malnutrition and disease will have impaired growth,
and these effects will still be captured by today's height.

The age-based z-scores are constructed on a weekly basis up to the age of 60 months, and on a monthly basis thereafer.


\subsubsection{Height-Based Z-scores}
The height-based z-scores are the Weight-for-Height Z-score (WHZ)
and Body Mass Index Z-score (BMIZ).
Where the BMI (or Quetelet) is a transformed version of the WHZ, which has a quadratic height effect.
The equation for BMI is:
\[
\textbf{BMI} = \frac{\textbf{weight(kg)}}{\textbf{height(m)}^2}
\]

These scores compare children with others of the same height, irrispective of their age.
As a results we only observe the relatively short-term effect of weight.
The height-based metrics thus provide is with a short-term insight.

The height-based z-scores are available on a semi-centimeter level throughout all heights.

\subsection{Data Structure}
The NIDS uses a file and data structure which is ill suied for panel data analysis.
We therefore transform the data to a format which is more conducive to our analysis.
In doing so, we try to stay as close as possible to the`Tidy data' structure, as described in \textcite{wickham2014tidy}.
This is easiest using the R package `Reshape2' by the same author \parencite[Reshape2 implementation]{wickham2007reshaping}.

\section{Research Design}
\label{sa:identification}
% The policy change, a natural experiment



\subsection{Ordinary Least Squares}
We first perform an Ordinary Least Squares estimation (OLS).
This is done using the linear model (lm) procedure from the R package `Base' \parencite{R}.

\subsection{Difference in Differences}
In order to fully exploit the available data and the policy change,
we also employ a `Difference-in-Differences' estimator (DiD).
This estimator operationalised by using the fixed-effects (within) estimator, with a time-effect\footnote{The time effect estimated here is symmetric to the individual effect.
We use the term `time effect', since this is a more meaningful description of the policy change.}.

We perform the estimations using the R package `PLM' \parencite[see][]{croissant2008panel}.
It is worth noting  that questions have been raised about the Difference-in-Difference estimator being employed in certain situations,
for example by (ironically) \textcite{bertrand2004much}.

\subsubsection{Formal DiD Specification}
% ATET etc.

\section{Results}
In \autoref{sa:hazwaz} we present our estimation results for the age-based z-scores,
and in \autoref{sa:whzbmiz} we present our estimation results for the height-based z-scores.

% LaTeX table created manually
% Fri Dec 27 23:22:00 2013
\begin{table}[h!]
\centering
\caption{Estimation Results: HAZ \& WAZ}
\label{sa:hazwaz}
\begin{tabular}{l|rrr|rrr}
% \hline
& \multicolumn{3}{c}{Height for Age Z-score} & \multicolumn{3}{c}{Weight for Age Z-score}\\
\hline
specification & 1 & 2 & 3 & 1 & 2 & 3\\
\hline
w\_spen\_m & 0.2366 & *0.8228 & 0.7908 & 0.2366 & 0.2981 & 0.4780 \\
w\_spen\_w & -0.2331 & 0.1053 & 0.1072 & -0.2331 & -0.3112 & -0.3280 \\
elig.men.60 & & **-0.3419 & **-0.3465 & & ***-0.3475 & **-0.3243 \\
w\_spen\_m1:elig.men.60 & & & 0.0446 & & & -0.2545 \\
% \hline
w\_h\_tinc & -0.0000 & -0.0000 & -0.0000 & -0.0000 & -0.0000 & -0.0000 \\
\end{tabular}
\end{table}

% LaTeX table created manually
% Fri Dec 27 23:53 2013
\begin{table}[h!]
\centering
\caption{Estimation Results: WHZ \& BMIZ}
\label{sa:whzbmiz}
\begin{tabular}{l|rrr|rrr}
& \multicolumn{3}{c}{Weight for Height Z-score} & \multicolumn{3}{c}{Body Mass Index Z-score}\\
\hline
specification & 1 & 2 & 3 & 1 & 2 & 3 \\
\hline
w\_spen\_m &  -0.3532 & -0.3210 & -0.4303 & *-0.8058 & *-0.7905 & *-1.0226 \\
w\_spen\_w & 0.0655 & 0.0371 & 0.0478 & -0.1592 & -0.1956 & -0.1742 \\
elig.men.60 & & -0.1417 & -0.1574 & & -0.1674 & -0.2049 \\
w\_spen\_m1:elig.men.60 & & & 0.1484 & & & 0.3407 \\
w\_h\_tinc & -0.0000 & -0.0000 & -0.0000 & -0.0000 & 0.0000 & 0.0000
\end{tabular}
\end{table}



\subsection{Robustness Checks}

\subsubsection{OLS Robustness}

\subsubsection{DiD Robustness}
% HAUSMAN TEST FOR FE / RE

\section{Conclusion}

\printbibliography
\end{refsection}