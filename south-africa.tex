\documentclass[draft.tex]{subfiles}

\begin{document}

\chapter{Additional South African Evidence on Pensions and Child Growth}

\section{Abstract}

In this paper I look at the effect of the gender of pension recipients
on the growth of children in the same households.

I do this by comparing z-scores of anthropometrics of South African
children living in the same household as state pension recipients. This
paper exploits the fact that the data set consists of two surveys which
were done before and after the lowering of the pension eligibility age
for men to the same age as women.

The main preliminary finding is that the household effect was only
significant in 2012.

\section{Introduction}

This paper looks at the effect of the gender of pension recipients on
the growth of children in their household in South Africa. The approach
is very similar to Duflo
(\href{http://www.jstor.org/discover/10.2307/117257}{2000},\href{http://wber.oxfordjournals.org/content/17/1/1}{2003}).
The deviation from international standards
\href{http://www.who.int/childgrowth/en/}{Child Growth Standards} for
weight-for-height and length-for-age are computed as z-scores. These are
then compared for different pension recipient status.

The anthropometrics are useful for computing z-scores. These z-scores
are considered a good representation of short-term or long-term
malnutrition respectively, especially for children between 6 and 60
months old.

The South African pension is a useful variable to measure income because
of its criteria. Besides a maximum level of income, the only criterium
is the age of a person. Because of this the status as a recipient is
quite exogenous and there are few selection bias problems. The
problematic difference between the eligibility age of men and women was
eliminated between the two surveys which creates an interesting natural
experiment.

Duflo finds evidence in South Africa's 1993
\href{http://microdata.worldbank.org/index.php/catalog/297}{Integrated
Household Survey} that girls' short term nutrition (weight-for-height)
was positively influence by living with the maternal grandmother if the
grandmother was eligible for a state pension. This is directly after the
significant increase in the pension sum for blacks. The
pension-eligibility age at this time was 60 for women and 65 for men.

This study deviates from the Duflo study in several ways. In 2008 and
2012 the first and second wave of the
\href{http://www.nids.uct.ac.za/home/}{National Income Dynamics Study}
have gathered similar data. In the period 2008-2010, the government
lowered the eligibility age for men from 65 to 60
(\href{http://www.southafrica.info/services/government/pension-160708.htm}{Announcement}).

This happened in a few steps. As of July 16th 2008 men aged 63 and 64
were eligible for a pension. Men aged 61 and 62 became eligible in April
2009. Finally in January 2010, pension eligibility age was at 60 for all
citizens.

Another deviation is the usage of
\href{http://www.who.int/childgrowth/en/}{Child Growth Standards} in
stead of CDC's, since these have superseded the CDC charts, however this
should not be of any consequence.

The main preliminary result is that the household effect (i.e.~having
one or more recipients in the household) is very insignificant in the
2008 estimate. But in the 2012 estimate it is very significant.

\section{Data}

In this paper I use data from two sources. The first is the South
African National Income Dynamics Survey and the second is the World
Health Organization's Child Growth Standards
\href{http://www.who.int/childgrowth/en/}{Child Growth Standards}.

The main source of data is the NIDS. This survey collects data on a
representative set of appproximately 10,000 South African households.
The primary information types I use are, the child anthropometrics, the
age and gender of household members, and the status as state pension
recipients.

For adults several variables measure the different amounts and sources
of income. Among those, a variable if the adult receives a state
pension, and if so, how much. This is a numeric variable, the values of
which lie very close together. For simplicity I have temporarily used
this variable as a dummy.

Children's anthropometrics are taken, these are length/height, weight,
and waist. Using these anthropometics and WHO growth standards, z-scores
have been calculated. Unfortunately wave 2 (2012) accidentally omitted
the z-scores, so that these cannot be evaluated until an updated version
is published. However, I have computed the length-for-age z-scores
manually.

In 2006 the WHO published its standards for child growth
\href{http://www.who.int/childgrowth/en/}{Child Growth Standards}. These
standards are based on the scores of children from different ethnic
populations in households which observed a healthy lifestyle. The
standards provide the means and standards deviations used. These are on
a monthly basis for height-for-age, and on a semi-centimeter level for
weight-for-height scores.

\texttt{\{r, include=FALSE\} load("w1w2.RData") library(sem) library(ggplot2)}

\texttt{\{r\} p1 \textless{}- ggplot(subset(w1\_child,w1\_woman\_60 == 1), aes(w1\_c\_age\_m, w1\_zhfa, colour=factor(w1\_c\_woman))) p1 + stat\_smooth(method="loess")\# + geom\_point()}

\section{Results}

Basic HAZ (2008). \texttt{\{r\} summary(w1\_haz0)}

HAZ 2008, with 2012 eligiblity as an IV
\texttt{\{r\} summary(tsls\_w1w2\_haz0)}

HAZ 2008, with 2012 eligiblity as an IV (men over 65 only)
\texttt{\{r\} summary(tsls\_w1w2\_haz1)}

\end{document}